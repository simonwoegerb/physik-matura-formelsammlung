% ! TeX root = *.tex
\documentclass{paper}
\usepackage[utf8]{inputenc}
\usepackage[german]{babel}
\usepackage{multicol}
\usepackage{color}
\usepackage{tabularx}
\usepackage[per-mode=fraction]{siunitx}

\usepackage{amsfonts,amssymb,amsthm,bm,amsmath}
\usepackage{esvect}
\makeatletter
\newcommand{\mathleft}{\@fleqntrue\@mathmargin0pt}
\newcommand{\mathcenter}{\@fleqnfalse}
\makeatother
{\renewcommand{\arraystretch}{1.4} %<- modify value to suit your needs
\begin{document}
\title{Formelsammlung}
\subtitle{Schriftliche Reifeprüfung Physik 2023/2024}
\author{Simon Wögerbauer}
\date{\today}
\maketitle
\smalltableofcontents
\mathleft
\section{Grundlegendes}
\begin{tabularx}{\textwidth}{X|X}
	Winkelgeschwindigkeit zu Frequenz und Periodendauer & $\omega = \frac{2\pi}{T}=2\pi*f$ \\
	Masse & $m\: [\unit{\kilo\gram}]$ \\
	Länge & $s\: [\unit{\meter}]$\\
Zeit & $t\:[s]$\\
Geschwindigkeit & $v\:[\unit{\meter\per\second}]$\\
Beschleunigung & $a\:[\unit{\meter\per\second\squared}]$\\
Frequenz & $f\: [\unit{\hertz}]$\\
Kraft & $F\: [\unit{\newton}]$\\
Arbeit & $W \: [\unit{\newton\meter}]$\\
Leistung & $P \: [\unit{\watt},\unit{\joule\per\second}]$
\end{tabularx}

\section{Kinematik}
\mathleft
\begin{tabularx}{\textwidth}{r|X}
	Newton II: & $F=m*a$\\
	Newton III: & ${F}_{\vv{\!AB}}=-{F}_{\vv{\!BA}}$,$\quad F_{AB} = F_{BA}$, actio = reactio \\
	Beschleunigung & $a=\frac{\Delta v}{\Delta t}$\\
	Zentripedalkraft & $F_{ZP}=\frac{mv^2}{2}$\\
	Potentielle Energie & $W_{pot} = F_G * h$\\
	Kinetische Energie & $W_{kin} = \frac{1}{2}*m*v^2$ \\
	Gravitationskraft & $F_G = \frac{m_1 * m_2 }{r^2}$\\
	Gewichtskraft & $F_g = m * g$\\
\end{tabularx}
\end{document}
