% ! TeX root = *.tex
\documentclass[11pt,a4paper]{paper}
\usepackage[utf8]{inputenc}
\usepackage[german]{babel}
\usepackage{multicol}
\usepackage{color}
\usepackage{tabularx}
\usepackage[margin=3cm]{geometry}
\usepackage[per-mode=fraction]{siunitx}

\usepackage{amsfonts,amssymb,amsthm,bm,amsmath}
\usepackage{esvect}
\makeatletter
\newcommand{\mathleft}{\@fleqntrue\@mathmargin0pt}
\newcommand{\mathcenter}{\@fleqnfalse}
\makeatother
{\renewcommand{\arraystretch}{1.4} %<- modify value to suit your needs
\begin{document}
\title{Formelsammlung}
\subtitle{Schriftliche Reifeprüfung Physik 2023/2024}
\author{Simon Wögerbauer}
\date{\today}
\maketitle
\smalltableofcontents
\mathleft
%\begin{multicols}{2}
\section{Grundlegendes}
\begin{tabularx}{\textwidth}{X|X}
	Winkelgeschwindigkeit zu Frequenz und Periodendauer & $\omega = \frac{2\pi}{T}=2\pi*f$ \\
	Masse & $m\: [\unit{\kilo\gram}]$ \\
	Länge & $s\: [\unit{\meter}]$\\
Zeit & $t\:[s]$\\
Geschwindigkeit & $v\:[\unit{\meter\per\second}]$\\
Beschleunigung & $a\:[\unit{\meter\per\second\squared}]$\\
Frequenz & $f\: [\unit{\hertz}]$\\
Kraft & $F\: [\unit{\newton}]$\\
Arbeit & $W \: [\unit{\newton\meter}]$\\
Leistung & $P \: [\unit{\watt},\unit{\joule\per\second}]$
\end{tabularx}

\section{Mechanik}
\mathleft
\begin{tabularx}{\textwidth}{X|X}
	Newton II: & $F=m*a$\\
	Newton III: & ${F}_{\vv{\!AB}}=-{F}_{\vv{\!BA}}$,$\quad F_{AB} = F_{BA}$, actio = reactio \\
	Beschleunigung & $a=\frac{\Delta v}{\Delta t}$\\
	Zentripedalkraft & $F_{ZP}=\frac{mv^2}{2}$\\
	Potentielle Energie & $W_{pot} = F_G * h$\\
	Kinetische Energie & $W_{kin} = \frac{1}{2}*m*v^2$ \\
	Gravitationskraft & $F_G = \frac{m_1 * m_2 }{r^2}$\\
	Gewichtskraft & $F_g = m * g$\\
	Dichte & $\delta = \frac{m}{V}$\\
	Geschwindigkeit & $v = \frac{\Delta s}{\Delta t}$\\
	Unbeschleunigte Bewegung & $ v = \frac{s}{t}$\\
	gleichmäßig beschleunigte Bewegung & $s = \frac{a}{2}*t^2$ \hspace{.5cm} $v = a * t $\\
	Hydrostatischer Druck & $p = \rho * g * h$ \\
	Bernoulli-Gleichung & $\frac{1}{2}*\rho*v^2 + \rho * g * h + p = const.$\\
	3. Kepler-Gesetz (Planetenbahnen) & $ (\frac{T_1}{T_2})^2 = (\frac{r_1}{r_2})^3$\\
	Impuls & $\vec{p} = m * \vec{v}$\\
\end{tabularx}
\section{Elektronik}
\begin{tabularx}{\textwidth}{X|X}
	Elektrische Stromarbeit & $ W_{el} = Q * I = I * U * t$\\
	Elektrische Stromleistung & $P_{el} = U * I$\\
	Elektrischer Strom & $I = \frac{\Delta Q}{\Delta t}$\\
	Ohm'sches Gesetz & $I=\frac{U}{R}$\\
	Widerstand Serie & $R_{ges} = R_1 + R_2 + R_n$\\
	Widerstand Parallel & $\frac{1}{R_{ges}} = \frac{1}{R_1} + \frac{1}{R_2} + \frac{1}{R_n}$\\
	Coulomb-Kraft & $F = \frac{1}{4*\pi * \epsilon_{0}} * \frac{Q_1 * Q_2}{r^2}$\\
	Ladung eines Kondensators & $Q = U * C$\\
	Wechselspannung & $U = U_0 * \sin{\omega t}$\\
\end{tabularx}
\section{Wärmelehre}
\begin{tabularx}{\textwidth}{X|X}
	Längenänderung & $l = l_0 * (1 + \alpha * \Delta T)$\\
	Allgemeine Gasgleichung & $pV = n*R*T = n *K_b * T$\\
	Isobare Zustandsänderung & $\frac{V}{T} = const.$\\
	Isotherme Zustandsänderung & $p*V = const.$\\
	Isochore Zustandsänderung & $\frac{p}{T} = const.$\\
	Adiabatische Zustandsänderung & $p * V^K = const$
\end{tabularx}
%\end{multicols}
\section{Schwingungen und Wellen}
\begin{tabularx}{\textwidth}{X|X}
Harmonische Schwingung / Elongation & $y = r * \sin{\omega t}$\\
Harmonische Schwingung / Geschwindigkeit & $y = r * \omega * \cos{\omega t}$\\
Harmonische Schwingung / Beschleunigung & $y = -r * \omega^2 * \sin{\omega t}$\\
Federpendel & $T = 2\pi * \sqrt{\frac{m}{k}}$\\
Fadenpendel & $T= 2\pi * \sqrt{\frac{l}{g}}$\\

\end{tabularx}
\end{document}

