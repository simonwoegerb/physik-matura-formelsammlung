% ! TeX root = *.tex
\documentclass[12pt,a4paper]{paper}
\usepackage[utf8]{inputenc}
\usepackage[german]{babel}
\usepackage{multicol}
\usepackage{color}
\usepackage{tabularx}
\usepackage[margin=3cm]{geometry}
\usepackage[per-mode=fraction]{siunitx}
\usepackage{amsfonts,amssymb,amsthm,bm,amsmath}
\usepackage{esvect}
\makeatletter
\newcommand{\mathleft}{\@fleqntrue\@mathmargin0pt}
\newcommand{\mathcenter}{\@fleqnfalse}
\makeatother
{\renewcommand{\arraystretch}{1.4} %<- modify value to suit your needs
\title{Formelsammlung}
\subtitle{Schriftliche Reifeprüfung Physik 2023/2024}
\author{}
\date{\today}
\begin{document}
\maketitle
\smalltableofcontents
\mathleft
%\begin{multicols}{2}
\section{Grundlegendes}
\begin{tabularx}{\textwidth}{X|X}
	Winkelgeschwindigkeit zu Frequenz und Periodendauer & $\omega = \frac{2\pi}{T}=2\pi*f$ \\
	Masse & $m\: [\unit{\kilo\gram}]$ \\
	Länge & $s\: [\unit{\meter}]$\\
Zeit & $t\:[s]$\\
Spezifische Wärmekapazität & $c\: [\unit{\joule\per\kilo\gram\per\kelvin}]$\\
Geschwindigkeit & $v\:[\unit{\meter\per\second}]$\\
Beschleunigung & $a\:[\unit{\meter\per\second\squared}]$\\
Frequenz & $f\: [\unit{\hertz}]$\\
Kraft & $F\: [\unit{\newton}]$\\
Arbeit & $W \: [\unit{\newton\meter}]$\\
Leistung & $P \: [\unit{\watt},\unit{\joule\per\second}]$\\
Elektrischer Widerstand & $R\:[\unit{\ohm}]$\\
Elektrische Spannung & $U\:[\unit{\volt}]$\\
Elektrische Stromspannung & $I\:[\unit{\ampere}]$\\
Ladung & $C\:[\unit{\coulomb}]$
\end{tabularx}

\section{Mechanik}
\mathleft

\begin{tabularx}{\textwidth}{X|X}
	Ableitungen & $s'(t) = v(t) \;\;\&\;\; v'(t) = a(t)$\\
	Newton II: & $F=m*a$\\
	Newton III: & ${F}_{\vv{\!AB}}=-{F}_{\vv{\!BA}}$,$\quad F_{AB} = F_{BA}$, actio = reactio \\
	Beschleunigung & $a=\frac{\Delta v}{\Delta t}$\\
	Zentripedalkraft & $F_{ZP}=\frac{mv^2}{r}$\\
	Potentielle Energie & $W_{pot} = F_G * h$\\
	Kinetische Energie & $W_{kin} = \frac{1}{2}*m*v^2$ \\
	Gravitationskraft & $F_G = \frac{m_1 * m_2 }{r^2}$\\
	Gewichtskraft & $F_g = m * g$\\
	Dichte & $\rho = \frac{m}{V}$\\
	Geschwindigkeit & $v = \frac{\Delta s}{\Delta t}$\\
	Unbeschleunigte Bewegung & $ v = \frac{s}{t}$\\
	Gleichmäßig beschleunigte Bewegung & $s = \frac{a}{2}*t^2$ \hspace{.5cm} $v = a * t $\\
	Hydrostatischer Druck & $p = \rho * g * h$ \\
	Bernoulli-Gleichung & $\frac{1}{2}*\rho*v^2 + \rho * g * h + p = const.$\\
	3. Kepler-Gesetz (Planetenbahnen) & $ (\frac{T_1}{T_2})^2 = (\frac{r_1}{r_2})^3$\\
	Impuls & $\vec{p} = m * \vec{v}$\\
	Lotrechter Wurf Höhe & $h = \frac{g* t_{ou}^2}{2}$\\
	Lotrechter Wurf Geschwindigkeit & $v = g * \Delta t$\\
	Schiefer Wurf Gleichung & $y(t) = -\frac{1}{2}*g*t^2 + v_0*sin(\alpha)*t + h_0$\\
	Schiefer Wurf Geschwindigkeiten & $v_{V} = v_0 * \sin{\alpha} \:\:\:\:\: v_{H} = v_0 * \cos\alpha$\\
\end{tabularx}
\section{Relativität \& Quantenphysik}
\begin{tabularx}{\textwidth}{X|X}
Zeitdilatation & $t' = t * \sqrt{1- (\frac{v}{c})^2}$\\
Längenkontraktion & $s' = s * \sqrt{1- (\frac{v}{c})^2}$\\
Dynamische Masse & $m_{d} = \frac{m_0}{\sqrt{1- (\frac{v}{c})^2}}$\\
Geschwindigkeitsaddition & u = $\frac{{u' + v}}{{1 + \frac{{u'  * v}}{{{c^2}}}}} $, wobei $u$\dots Geschw. im System $S$, $u'$\dots Geschw. im System $S'$, $v'$\dots Geschw. von $S'$ im Bezug zu $S$\\
Heisenberg'sche Unschärferealation & $\Delta x \cdot \Delta p \gtrapprox h $

\end{tabularx}
\section{Elektronik}
\begin{tabularx}{\textwidth}{X|X}
	Elektrische Stromarbeit & $ W_{el} = Q * I = I * U * t$\\
	Elektrische Stromleistung & $P_{el} = U * I$\\
	Elektrischer Strom & $I = \frac{\Delta Q}{\Delta t}$\\
	Ohm'sches Gesetz & $I=\frac{U}{R}$\\
	\textbf{Serienschaltung} Widerstand & $R_{ges} = R_1 + R_2 + \dots + R_n$\\
	\textbf{Serienschaltung} Spannung & $U_{ges} = U_1 + U_2 + \dots + U_n$\\
	\textbf{Serienschaltung} Stromstärke & $ I_{ges} = I_1 = I_2 = \dots  = I_n$\\

	\textbf{Parallelschaltung} Widerstand & $\frac{1}{R_{ges}} = \frac{1}{R_{1}} + \frac{1}{R_{2}} + \dots + \frac{1}{R_{n}}$\\
	\textbf{Parallelschaltung} Spannung & $U_{ges} = U_{1} = U_{2} = U_{n}$\\
	\textbf{Parallelschaltung} Stromstärke  & $I_{ges}=I_{1}+I_{2}+\dots+I_{n}$\\
	Coulomb-Kraft & $F = \frac{1}{4*\pi * \epsilon_{0}} * \frac{Q_1 * Q_2}{r^2}$\\
	Ladung eines Kondensators & $Q = U * C$\\
	Wechselspannung & $U = U_0 * \sin{\omega t}$\\
\end{tabularx}
\section{Thermodynamik}
\begin{tabularx}{\textwidth}{X|X}
	Wärmeenergie & $Q=c*m*\Delta T$\\
	Längenänderung & $l = l_0 * (1 + \alpha * \Delta T)$\\
	Allgemeine Gasgleichung & $pV = n*R*T = n *K_b * T$\\
	Isobare Zustandsänderung & $\frac{V}{T} = const.$\\
	Isotherme Zustandsänderung & $p*V = const.$\\
	Isochore Zustandsänderung & $\frac{p}{T} = const.$\\
	Adiabatische Zustandsänderung & $p * V^{1.4} = const$
\end{tabularx}
%\end{multicols}
\section{Schwingungen und Wellen}
\begin{tabularx}{\textwidth}{X|X}
Harmonische Schwingung / Elongation & $y = r * \sin{\omega t}$\\
Harmonische Schwingung / Geschwindigkeit & $y = r * \omega * \cos{\omega t}$\\
Harmonische Schwingung / Beschleunigung & $y = -r * \omega^2 * \sin{\omega t}$\\
Federpendel & $T = 2\pi * \sqrt{\frac{m}{k}}$\\
Fadenpendel & $T= 2\pi * \sqrt{\frac{l}{g}}$\\
\end{tabularx}
\end{document}

